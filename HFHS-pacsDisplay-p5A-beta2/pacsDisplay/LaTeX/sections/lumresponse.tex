%\documentclass[../readme.tex]{subfiles} 
%\begin{document}

\section{Luminance Response Measurements}
\label{sec:lumresponsemeasurements}

The lumResponse application can be used to measure the gray palette for a display and record the luminance values in a uLR text file (Palette modes). These uLR palettes are then used to generate a DICOM calibration LUTs as described in the next section. LumResponse can also be used to measure the luminance response of a calibrated monitor and record the values in a cLR text file (QC modes). The cLR values can then be evaluated in relation to the DICOM GSDF. 

\subsection{LumResponse}

This program puts a large secondary window on the screen with a gray background and a central square target region. The target gray level is varied to assess display luminance response. The gray intensity of the target region is cycled through increasing intensity values to measure luminance versus gray value. The measurements are made by luminance meter (photometers or colorimeters), that are typically connected using a USB port. Currently, four meter types are supported. 

\subsubsection{Mode Options}

The 'Mode' button at the top left of the utility selects the type of measurement to be made. The current mode is displayed in the window title. The various modes are as follows: 

For uLR measurement: 

\textbf{1786 Palette} This mode is used for measuring the uncalibrated luminance response of a color display. Make sure the display is using a linear LUT (i.e - no calibration) before proceeding. 

\textbf{766 Palette} This mode is used for measuring the uncalibrated luminance response of a mono\-chrome display. Make sure the display is using a linear LUT (i.e - no calibration) before proceeding. 

For cLR measurement: 

\textbf{QC (256x1)} This mode is used for measuring the luminance response of a display after a LUT has been installed in order to verify that a DICOM calibration is in place. One measurement is made for 256 driving levels. 

\textbf{QC (16x2)} This mode is used for measuring the luminance response of a display after a LUT has been installed in order to verify that a DICOM calibration is in place. Two measurements are made at each of 16 different primary driving levels (8,24,40,...248). The measurements are made a driving levels just above and just below the primary driving level from which the contrast is computed. For some luminance meters, color gray tracking is also reported. 

Other modes: 

\textbf{DEMO MODE} This mode is for demonstration purposes. It will quickly cycle from black to white but no measurements will be taken. The IL1700 does not need to be connected for the demo to run. 

\textbf{OTHER} This option allows the user to use a customized perturbation series for the luminance measurements. For advanced users only. 

\subsubsection{Geometry and Meter Options}

The "GEOM" and "METER" buttons provide access to advanced options for the test image window and the photometer, respectively. The "GEOM" options can be used to customize the geometry and position of the test image. The initial values for the geometry options are set in the LRconfig.txt file which is in the \textnormal{.../pacsDisplay-BIN/lumResponse/} directory of Program Files [(x86)]. The initial configuration is for a 1024x1024 window with an 8 cm gray target region on monitors with a .230 mm pixel pitch. This should be suitable for the majority of monitor models. 

The "METER" options include selection of the meter model and settings effecting the way display luminance is measured. Further information is given by the "?" buttons next to each setting. 

Four meters are currently supported: 

\textbf{International Light IL1700 Research Radiometer} This device uses a serial line interface to report measured values. It has been replaced by the ILT1700 that uses a USB interface. The IL1700 option is retained only for legacy purposes. 

\textbf{IBA Dosimetry LXcan Spot Luminance Meter} This device is capable of making spot luminance measures some distance from the monitor surface. It is effective in measuring the amient luminance which is used in building a calibration LUT. A USB driver provided by IBA Dosimetry is used to record values. 

and two meters that use the Argyll CMS spotread procedure (www.argyllcms.com):

\textbf{X-Rite i1Display 2 photometer} This modestly priced photometer uses a USB interface. The photometer used the USB device driver from the open source Argyll Color Management System (Argyll CMS, www.argyll.com) which must be loaded the first time the meter is used. (see \href{www.argyllcms.com/doc/Installing_MSWindows.html}{www.argyllcms.com/doc/Installing\_MSWindows.html}) While many of these are still in use, it has been replaced by the i1Display Pro described next. 

\textbf{X-Rite i1Display Pro} This recently introduced and modestly priced photometer from X-Rite has improved precision and faster response than the i1Display 2. It is the currently recommended device if you need to buy a photometer and is available from several distributors for about \$200-250 USD. It is also sold with different labels such as the NEC SpectraSenor Pro. The device uses communicates using USB ports as a human interface device (HID) using the Windows HID driver. As such, it will generally attach and be ready to use without the need to load a driver.

\subsubsection{Making Measurements} 

Once a mode has been selected and the meter specified (defaults to i1Display Pro), a Display ID should be created to identify the measurement results once they are saved. The name that is selected will be used to name the output file. 

The "Display \#'' box is used to select the display that you wish to identify. This number is the same as that used by Windows and listed in the Display Properties window. Pressing the "GET EDID ID'' button will then retrieve the model and serial number (S/N) information from the EDID for that display and use it to build the Display ID. The result is displayed in the text box. A custom ID can be entered into the text box if so desired. It is suggested that the model and serial number of the display be included. 

The remaining portion of the lumResponse window provides the steps for the luminance response measurement and a button to begin each step. They are described here: 

\paragraph{Step 1: Position Test Image} This opens up a test image window that needs to be centered on the screen of the display to be measured. The window should be about the same size as the screen. The display dize can be changed manually using the GEOM settings. 

\paragraph{Step 2: Initialize Meter} This button begins communication with the luminance meter. It activates the meter display at the bottom of the window, providing information regarding the measurement. The details of this monitor are explained further below.

At this point, the luminance meter should be setup in front of the display, and centered on the square target in the middle of the test window. Depending on the meter used, it should be positioned close to or in contact with the display surface such that the luminance is not perturbed. the room lights should be set low and if necessary the monitor should be covered with a dark cloth. If used, the cloth must not cover any vents in the back of the display as this can cause it to heat up quickly and may affect the measurement. 

\paragraph{Step 3: Record Data} This starts the measurement process. You will be asked to be sure that the monitor is in the correct calibration (linear for uLRs and DICOM for cLRs) and that the power saver settings will not cause the screen to darken. Pressing this button once a measurement has begun will pause the measurement and provide an option to abort or continue. 

\paragraph{Step 4: Save Data Press} this button to save the luminance data once a measurement is complete. You will be asked where to save the output file. The default directory in the \_NEW folder in the LUT-Library. An option to plot the luminance vs p-values is provided after the data is saved.
%Should this last step be updated to include what is plotted for the QC run? 

Columns for the luminance response saved file:

%\begin{center}
%\begin{tabular}{rp{.6\textwidth}}
%\toprule
%MEASUREMENT No. & Sequential measurment number.\\
%\\
%AVG LUMINANCE & The average luminance value measured for each gray step.\\
%\\
%RGB VALUE & Octal RGB value for the measurement.\\
%\\
%GRAY-STEP & The current stage of the luminance measurement. This number represents the standard graylevel (1-256). Negative values represent the initialization frames.\\
%\\
%SUB-STEP & This second number represents the perturbation steps in the graylevel sequence. There are 7 sub- steps in 1786 mode and 3 in 766 mode. The 256 and Demo modes do not use sub-steps.\\
%\\
%dL/L & Relative difference between the current and the prior measurement.\\
%\\
%x & Chromaticity x value (for supported meters, Yxy, CIE 1932)\\
%\\
%y & Chromaticity y value (for supported meters, Yxy, CIE 1932)\\
%\bottomrule
%\end{tabular}
%\end{center}

\begin{itemize}
	\item[] \textbf{MEASUREMENT No.} \enskip Sequential measurement number.
	\item[] \textbf{AVG LUMINANCE} \enskip The average luminance value measured for each gray step.
	\item[] \textbf{RGB VALUE} \enskip Octal RGB value for the measurement.
	\item[] \textbf{GRAY-STEP} \enskip The current stage of the luminance measurement. This number represents the standard graylevel (1-256). Negative values represent the initialization frames.
	\item[] \textbf{SUB-STEP} \enskip This second number represents the perturbation steps in the graylevel sequence. There are 7 sub- steps in 1786 mode and 3 in 766 mode. The 256 and Demo modes do not use sub-steps.
	\item[] \textbf{dL/L} \enskip Relative difference between the current and the prior measurement.
	\item[] \textbf{x} \enskip Chromaticity x value (for supported meters, Yxy, CIE 1932)
	\item[] \textbf{y} \enskip Chromaticity y value (for supported meters, Yxy, CIE 1932)
\end{itemize}

Files are save with files names of:
\ind{\textnormal{uLR\_MANF\_MODEL\_SN.txt} for uncalibrated palette files and\\
\textnormal{cLR\_MANF\_MODEL\_SN.txt} for calibrated QC files.} 

The luminance response recorded with lumResponse represent surface luminance in the absence of the luminance caused by reflected room lights, Lamb. When doing QC evaluations or computing DICOM calibration LUTs, a value of Lamb is specified and added to each value. 

\subsection{QC Evaluations}

The American Association of Physicists in Medicine (AAPM) described a quantitative test of a DICOM calibrated monitor in 2005 (AAPM On Line report \#3). This was included in an IEC standard as a basic test (IEC). 

\subsubsection{QC (16x2) Evaluation}

The lumResponse applications includes routines to evaluate the QC 16x2 cLR. The AAPM and IEC method makes 18 luminance measurements at equally spaced gray levels (Digital Driving Levels, DDL). For a graphics system with 256 gray levels, these are spaced every 15 gray levels (17 x 15 + 1 = 256). 17 relative luminance changes are then evaluated and this contrast is compared to the DICOM GSDF. The ACR-AAPM-SIIM Technical Standard for Electronic Imaging (2012) recommends
\begin{quote}
"The contrast response of monitors used for diagnostic interpretation should be within 10\% of the GSDF over the full LR. For other uses, the contrast response should be within 20\% of the GSDF over the full LR." 
\end{quote}

The measurement protocol used by lumResponse for the QC 16x2 mode is essentially the same as the AAPM and IEC method except that 16 pairs of measures are made over the full range of a 256 level grayscale. The base value for each of the 16 measurement pairs increments in steps of 16; 0,16,32,...,240. The two measures are then made by adding 5 or 11 to the base value with the red, green, and blue values of each being equal. 

The contrast evaluated from these measure represents the contrast estimate, dL/L, for levels of 8, 24, 30, 46, ..., 248 with the contrast computed for gray level changes of 6. This provides an improved estimate of contrast compared to measures using gray level changes of 15 for the AAPM and IEC method. Additionally, the protocol makes three measurements of Lmin at a gray level of 0 and one measurement of Lmax at a gray level of 255. 

\textbf{Note:} The protocol is set in the 16phase.txt file located in the following folder in Program Files [(x86)]:
\begin{center}
	\textnormal{.../pacsDisplay-BIN/lumResponse/}
\end{center}


After saving the QC 16x2 cLR in a directory with the monitor model name, the user has the option to evaluate the results. An evaluation report, \textnormal{QC-lr.txt}, is place in the same folder along with four graphic plots in png format. The gnuplot command file is along left in the folder, \textnormal{QC-plot.gpl}. The plotted results include: 

\ind{
\textbf{QC-Plot-LUM.png} \enskip Luminance vs Gray Level using a semi log plot

\textbf{QC-Plot-dLL.png} \enskip Contrast, dL/L, vs Gray Level with 10\% and 20\% error conditions. The maximum relative error along with L'max and L'min are labeled.

\textbf{QC-Plot-JND.png} \enskip JNDs per Gray Level vs Gray Level (see DICOM 3.14)

\textbf{QC-Plot-uv.png} \enskip Color gray tracking with u'V' relative to D65.
}

The evaluation uses a default value of Lamb from the 'METER' options. The results can be re-evaluated with a different Lamb by using the QC check application that select the cLR file to re-evaluate and provides an entry for the new Lamb value.

\subsubsection{QC (256x1) Evaluation}

AAPM OR3 also described an advanced measurement test for which the calibrated luminance is measure for each of 256 gray levels and the contrast or JNDs per gray level evaluated using the difference between each measurement as adjacent gray levels. The QC (256x1) mode measures each of 256 luminances with red, green, and blue values being equal (i.e. 256 gray levels). These are then evaluated using a method similar to that for the QC (16x2) mode.


%\end{document}